\documentclass[a4paper,11pt]{article}
% Símbolo del euro
\usepackage[gen]{eurosym}
% Codificación
\usepackage[utf8]{inputenc}
% Idioma
\usepackage[spanish]{babel} % English language/hyphenation
\selectlanguage{spanish}
% Hay que pelearse con babel-spanish para el alineamiento del punto decimal
\decimalpoint
\usepackage{dcolumn}
\newcolumntype{d}[1]{D{.}{\esperiod}{#1}}
\makeatletter
\addto\shorthandsspanish{\let\esperiod\es@period@code}
\makeatother

\usepackage[usenames,dvipsnames]{color} % Coloring code

\usepackage{csvsimple}
\usepackage{adjustbox}
\newsavebox\ltmcbox


% Para matrices
\usepackage{amsmath}
\everymath{\displaystyle}

% Símbolos matemáticos
\usepackage{amssymb}
\let\oldemptyset\emptyset
\let\emptyset\varnothing

% Hipervínculos
\usepackage{url}

\usepackage[section]{placeins} % Para gráficas en su sección.
\usepackage{mathpazo} % Use the Palatino font
\usepackage[T1]{fontenc} % Required for accented characters
\newenvironment{allintypewriter}{\ttfamily}{\par}
\setlength{\parindent}{0pt}
\parskip=8pt
\linespread{1.05} % Change line spacing here, Palatino benefits from a slight increase by default

% Code for bash
\usepackage{minted}

% Imágenes
\usepackage{graphicx}
\usepackage{float}
\usepackage{caption}
\usepackage{wrapfig} % Allows in-line images



% Márgenes
\usepackage{geometry}
 \geometry{
 a4paper,
 total={210mm,297mm},
 left=30mm,
 right=30mm,
 top=25mm,
 bottom=25mm,
 }


% Referencias
\usepackage{fncylab}
\labelformat{figure}{\textit{\figurename\space #1}}

\usepackage{hyperref}
\hypersetup{
  colorlinks   = true, %Colours links instead of ugly boxes
  urlcolor     = blue, %Colour for external hyperlinks
  linkcolor    = blue, %Colour of internal links
  citecolor   = red %Colour of citations
}


\makeatletter
\renewcommand{\@listI}{\itemsep=0pt} % Reduce the space between items in the itemize and enumerate environments and the bibliography
\newcommand{\imagent}[4]{
    \begin{wrapfigure}{#4}{0.7\textwidth}
    \begin{center}
    \includegraphics[width=0.7\textwidth]{#1}
    \end{center}
    \caption{#3}
    \label{#4}
  \end{wrapfigure}
}


\newcommand{\imagen}[4]{
  \begin{minipage}{\linewidth}
    \centering
    \includegraphics[width=#4\textwidth]{#1}
    \captionof{figure}{#2}
    \label{#3}
  \end{minipage}
}

\newcommand{\imgn}[3]{
  \begin{minipage}{\linewidth}
    \centering
    \includegraphics[width=#3\textwidth]{#1}
    \captionof{figure}{#2}
  \end{minipage}
}

% Ejemplo de parámetro: ILS.r
\newcommand{\hrefr}[1]{
\href{../bin/#1}{#1}
}

%Customize enumerate tag
\usepackage{enumerate}
%Sections don't get numbered
%\setcounter{secnumdepth}{0}
\newcommand{\blue}[1]{\textcolor{blue}{#1}}
\begin{document}
%%%%%%%%%%%%%%%%%%%%%%%%%%%%%%%%%%%%%%%%%
% University Assignment Title Page
% LaTeX Template
% Version 1.0 (27/12/12)
%
% This template has been downloaded from:
% http://www.LaTeXTemplates.com
%
% Original author:
% WikiBooks (http://en.wikibooks.org/wiki/LaTeX/Title_Creation)
% Modified by: NCordon (https://github.com/NCordon)
%
% License:
% CC BY-NC-SA 3.0 (http://creativecommons.org/licenses/by-nc-sa/3.0/)
%
% Instructions for using this template:
% This title page is capable of being compiled as is. This is not useful for
% including it in another document. To do this, you have two options:
%
% 1) Copy/paste everything between \begin{document} and \end{document}
% starting at \begin{titlepage} and paste this into another LaTeX file where you
% want your title page.
% OR
% 2) Remove everything outside the \begin{titlepage} and \end{titlepage} and
% move this file to the same directory as the LaTeX file you wish to add it to.
% Then add \input{./title_page_1.tex} to your LaTeX file where you want your
% title page.
%
%%%%%%%%%%%%%%%%%%%%%%%%%%%%%%%%%%%%%%%%%
\begin{titlepage}

\newcommand{\HRule}{\rule{\linewidth}{0.5mm}} % Defines a new command for the horizontal lines, change thickness here

\center % Center everything on the page

%----------------------------------------------------------------------------------------
%	HEADING SECTIONS
%----------------------------------------------------------------------------------------
\textsc{\LARGE Universidad de Granada}\\[1.5cm]
\textsc{\Large Ingeniería del conocimiento}\\[0.5cm]

%----------------------------------------------------------------------------------------
%	TITLE SECTION
%----------------------------------------------------------------------------------------
\bigskip
\HRule \\[0.4cm]
{ \huge \bfseries SE inversor de bolsa}\\[0.4cm] % Title of your document
\HRule \\[1.5cm]

%----------------------------------------------------------------------------------------
%	LOGO SECTION
%----------------------------------------------------------------------------------------

\begin{center}
\includegraphics[width=8cm]{./data/ugr.jpg}
\end{center}
%----------------------------------------------------------------------------------------

\begin{minipage}{\textwidth}
\begin{center} \large
Ignacio Cordón Castillo, 25352973G\\
\url{nachocordon@correo.ugr.es}\\
\ \\
$4^{\circ}$ Doble Grado Matemáticas Informática\\
\end{center}
\end{minipage}


\vspace{\fill}% Fill the rest of the page with whitespace
\large\today
\end{titlepage}

\newpage
\tableofcontents
\newpage
% Examples of inclussion of images
%\imagent{ugr.jpg}{Logo de prueba}{ugr}
%\imagen{ugr.jpg}{Logo de prueba}{ugr2}{size relative to the \textwidth}

\section{Resumen funcionamiento}
La obtención de los datos de la bolsa puede hacerse ejecutando:

\begin{minted}{bash}
 ./scrapeo.py
\end{minted}

disponible en la carpeta \texttt{bin}.

Este archivo se conecta a varias web de interés económico con información disponible sobre la bolsa 
(\href{http://www.bolsamadrid.es}{Bolsa de Madrid}, \href{http://www.eleconomista.es}{El Economista},
\href{http://www.invertia.com/}{Invertia}) para \textit{scrapear} datos de interés (precio por acción,
per, rpd, tamaño de la empresa, variación mensual, trimestral, semestral, anual, capitalización, \ldots) 

Asimismo también efectúa un \textit{scrapeo} de datos de sectores (Bienes, Tecnología, Servicios, \ldots).

Los datos son guardados en los archivos \texttt{Analisis.txt} y \texttt{AnalisisSectores.txt} respectivamente,
disponibles en la carpeta \texttt{data}.

Los archivos de Clips(archivos \texttt{.clp} se encuentran en la carpeta \texttt{clp}.

Por clarificar un poco la estructura de ficheros de la práctica, tenemos:
\begin{itemize}
 \item \textbf{bin}: Archivos de \textit{scrapeo} de datos, y diccionarios con sectores correspondientes a cada empresa 
 del Ibex35 (\texttt{sectores.py}) y denominaciones usadas por cada empresa en las distintas páginas de las que se
 extraen datos (\texttt{empresas.py}).
 \item \textbf{data}: Archivos \texttt{.txt} de entrada al sistema (\texttt{Cartera}, \texttt{Analisis}, 
 \texttt{AnalisisSectores} y \texttt{Noticias}).
 \item \textbf{clp}: Archivos correspondientes a Clips.
\end{itemize}


El sistema carga, por este orden los siguientes módulos, a través del arvhivo \texttt{main.clp}:

\begin{itemize}
  \item \textbf{Módulo de estructura de datos}: carga los templates necesarios para el funcionamiento de la práctica: valores de cartera,
  valores de sectores, noticias, templates para definir valores inestables, peligrosos, infravalorados, sobrevalorados,
  estructura de datos de acciones propuestas \ldots

  \item \textbf{Módulo de lectura, y cálculo de valores inestables}: lee los archivos de cartera, sectores y noticias, y
  efectúa un cálculo de los valores inestables en función de ello.

  \item \textbf{Módulo de cálculo de valores peligrosos}
  
  \item \textbf{Módulo de cálculo de valores sobrevalorados}
  
  \item \textbf{Módulo de cálculo de propuestas}: declara todos los hechos en base de datos correspondientes a cálculo de
  propuestas posibles sobre valores en cartera (para vender), o valores que podrían adquirirse o intercambiarse por alguno en cartera.
  
  \item \textbf{Módulo menú}: muestra al usuario las 5 mejores propuestas con su rendimiento esperado y despliega un menú 
  sobre cuál de dichas propuestas llevar a cabo. En función de la opción escogida, se procesará una venta, una compra, o un
  intercambio de valores (se venden las acciones de una empresa en cartera para adquirir el equivalente en dinero de una
  empresa cuyas acciones son potencialmente revalorizables).
\end{itemize}

Una vez cargado el último módulo se limpian las propuestas calculadas, así como otras variables auxiliares para la ejecución
del menú, y se dispara de nuevo el módulo de cálculo de propuestas hasta que escojamos en el menú una opción distinta a
llevar a cabo alguna de las propuestas.

\section{Proceso de desarrollo}

\subsection{Sesiones con el experto}
  
\subsubsection{Primera sesión: adquisión de un conocimiento general sobre el sistema a desarrollar}

Se realiza una entrevista abierta al experto en la que se le plantean cuestiones tales como ¿cuál es el objetivo del sistema?,
¿qué entradas espera obtener?, ¿qué salidas espera obtener?, ¿de dónde se obtienen los valores necesarios para el sistema?,
¿cuál es el esquema general de funcionamiento del sistema?.

La información recopilada ha sido que buscamos un sistema que permita realizar simulaciones a partir de inversiones en bolsa,
al que se le proporciona una cartera de valores que poseemos, y que debe aportar a la salida dicha cartera modificada. 
Los datos es preciso obtenerlos del Ibex35 de forma gratuita, es decir, que no podemos pagar por la información que obtengamos.
El sistema debe aportar la información suficiente para que un aprendiz pueda extraer conocimiento sobre la bondad de las acciones
llevadas a cabo. Los valores pueden ser inestables, peligrosos, infravalorados y sobrevalorados y efectuaremos
una serie de propuestas a raíz de la información disponible sobre cada valor.

\subsubsection{Segunda sesión: cálculo de valores inestables}

Se plantea una entrevista estructurada al experto, efectuándole una relación de cuestiones. En adelante en todas las sesiones
se plantean entrevistas estructuradas.

\begin{itemize}
 \item ¿Qué tipos de noticias podemos tener? ¿Sobre sectores también podemos tener noticias?
  ¿Podemos tener noticias sobre la economía en general?
 \item ¿Qué se considera un valor inestable?
 \item ¿En qué casos un valor empieza a ser inestable? ¿Durante cuánto tiempo?
 \item ¿Puede un valor inestable dejar de serlo? ¿En qué casos? ¿Puede poner un ejemplo?
\end{itemize}

La información educida, a grandes rasgos en esta y posteriores sesiones sobre los valores que necesitam, ha sido la listada a continuación: 
Una empresa se considera inestable por defecto si pertenece al sector de la construcción. Los valores del sector servicios
se consideran inestables si la economía está bajando. Si un valor tiene una noticia negativa, se considera inestable durante 2 días.
Si tenemos una noticia negativa sobre un sector, todos los valores de ese sector pasan a ser inestables durante dos días.
Si hay una noticia negativa sobre la economía en general, todos los valores pasan a ser inestables.
Si hay una noticia positiva sobre un valor este deja de ser inestable 2 días.


\subsubsection{Tercera sesión: cálculo de valores peligrosos}

Cuestiones planteadas al experto:
\begin{itemize}
 \item ¿Cuándo un valor se considera que es peligroso?
 \item ¿Qué significa que su sector ha variado un 5\%(o más) que el valor? ¿Puede poner un ejemplo?
\end{itemize}

La información educida ha sido: un valor se considera peligroso si es inestable y está perdiendo de forma continua durante
los tres últimos días, o bien si ha estado perdiendo de forma continua durante los últimos 5 días y la variación en esos 5
días respecto a la del sector ha sido menor que -5\%, esto es $$var\_valor - var\_sector < - 5 $$

\subsubsection{Cuarta sesión: cálculo de valores sobrevalorados e infravalorados}

Cuestiones planteadas al experto:
\begin{itemize}
 \item ¿Cuándo un valor se considera que es sobrevalorado? ¿Y peligroso?
 \item ¿Cuándo el PER es alto? ¿Y mediano? ¿Y bajo?
 \item Ídem para el RPD.
 \item ¿Qué significa que una empresa es grande, mediana o pequeña?
 \item ¿Qué significa en el cálculo de valores infravalorados que una empresa haya subido pero no mucho en el último mes?
 \item ¿Qué significa en el cálculo de valores infravalorados que una empresa se comporte mejor que su sector?
\end{itemize} 

La información educida ha sido: 
\begin{enumerate}[$\rightarrow$]
 \item Una empresa está sobrevalorada si el PER es alto y el RPD bajo.
 \item En el caso de empresas pequeñas, están sobrevaloradas si el PER es alto; o el PER es mediano y el RPD es bajo.
 \item En empresas grandes, si el RPD es bajo y el PER mediano o alto, la empresa está sobrevalorada; si el RPD es mediano
 y el PER es alto, también.
 \item Una empresa está infravalorada si el PER es bajo y su RPD alto.
 \item Si una empresa ha caído más de un 30\%, ha subido, pero no mucho en el último mes (esto significa menos de un 10\%),
 y el PER es bajo, la empresa está infravalorada.
 \item Si una empresa es grande, su RPD alto, y su PER mediano, no está bajando y se comporta mejor que su sector (esto es
 la diferencia de variación de la cotización de la empresa en los últimos 5 días es superior a la variación del sector en 
 esos 5 días), la empresa está infravalorada.
 \item El tamaño de una empresa se mide por el porcentaje de capitalización que representa respecto al total del Ibex35. Si
 este porcentaje es menor que 2\%, la empresa es pequeña. Si está entre 2\% y 5\%, la empresa es mediana.; si es mayor que 
 un 5 \%, la empresa es grande.
 \item El PER de una empresa es bajo si es menor de 12\%, medio si está entre 12\% y 18\%. Alto si es superior al 18\%.
 \item El RPD de una empresa es bajo si es menor que un 2\%, medio si se encuentra entre un 2\% y un 5\%. Alto si es superior
 a un 5\%.
\end{enumerate}

\subsubsection{Quinta sesión: realización de propuestas}

Cuestiones planteadas al experto:
\begin{itemize}
 \item ¿Cuándo se invierte en acciones? ¿Cuánto dinero invertimos cuando lo hacemos?
 \item ¿Cuándo se venden acciones?
 \item ¿Cuántas propuestas debemos ofrecer? ¿En base a qué ordenamos las propuestas?
 \item ¿Cuál es el orden general en que debe realizarse todo el proceso?
\end{itemize} 

La información educida ha sido:

\begin{enumerate}[$\rightarrow$]
 \item Se venden acciones cuando una empresa es peligrosa, ha bajado el último mes y no ha bajado más de un 30\% respecto 
 a su sector el último mes.
 \item Se ofrecen propuestas conforme a un cálculo de rendimiento esperado. En el caso anterior por ejemplo, el rendimiento
 esperado era 20-rpd.
 \item Asimismo, también se venden acciones de empresas sobrevaloradas, cuyo rendimiento por año (cuyo concepto
 le es cuestionado, y responde que el rendimiento por año y nos contesta que se trata del rendimiento por dividendos más la
 variación anual) es menor que 5 más el precio del dinero (el declarado por el BCE).
 \item El rendimiento esperado en el caso anterior sería $$-RPD + 100 \frac{PER - PER\_SECTOR}{5*PER}$$
 \item Se efectúan compras cuando una empresa está infravalorada y el usuario tiene dinero para invertir. El rendimiento esperado
 es $$RPD + 100 \frac{PER\_SECTOR - PER}{5*PER}$$. Se invierte solo la mitad del dinero.
 \item También se efectúan compras cuando tengo una empresa en mi cartera no infravalorada y otra que no está sobrevalorada
 y cuyo RPD es mayor que el rendimiento por año + RPD + 1 de la empresa de mi cartera, entonces se cambian las acciones de una
 empresa por las de otra.
 \item Asimismo, el experto también ha proporcionado una explicación de por qué se realiza cada acción.
 \item Se ofrecen las 5 propuestas cuyo RE sea mayor, y se lleva a cabo una. Después el sistema volverá a ofrecer nuevas propuestas.
 \item El orden en que debe efectuarse el proceso es: lectura de datos, después cálculo de valores peligrosos, tras ello, cálculo de valores sobrevalorados
 e infravalorados a la vez, después cálculo de propuestas y por último ofrecimiento de propuestas, realimentándose estas dos
 últimas etapas.
\end{enumerate}


\subsection{Procedimiento de validación y verificación}

Además de los casos de uso típicos del funcionamiento de cada uno de los módulos, tales como:
\begin{itemize}
 \item La lectura de valores, sectores y noticias se realiza de forma comedida.
 \item Los valores inestables se marcan adecuadamente.
 \item Los valores peligrosos se marcan adecuadamente.
 \item Los valores sobrevalorados e infravalorados se marcan adecuadamente.
 \item Las propuestas se calculan conforme a lo descrito.
 \item Se efectúa lo pedido para cada una de los tipos de propuestas.
 \item El menú y el cálculo de propuestas se retroalimentan correctamente.
 \item \ldots
\end{itemize}

cuya corrección se ha comprobado mediante la ejecución de pruebas y visualización de los hechos generados en la base de
conocimiento de Clips, mediante la orden \texttt{(facts)}.

se ha hecho énfasis en que las conclusiones obtenidas en cada módulo, especialmente en el último, hayan sido correctas. Asimismo,
se ha buscado que el sistema sea lo más comprensible posible, a lo que ha colaborado la separación en módulos-archivos hecha, y
que la lógica de ordenación del disparado de reglas sea lo más simple posible.

El SE presenta medios de de explicación, pues cada opción ofertada se fundamenta, cubre el funcionamiento básico que se pedía
y permite añadir nuevo conocimiento de manera sencilla (basta actualizar la cartera y/o los datos de empresas y sectores).

En cuanto a la verificación, se ha comprobado que el sistema era preciso (revisando la sintaxis de programación de la lógica
de la base de datos), y consistente. En lo referente a la consistencia, se encontraron inconsistencias de tipo lógica en la
deducción de valores peligrosos, puesto que el experto aportó como regla que si un valores estaba perdiendo durante los
tres últimos días, se marcaba como peligroso. Y si estaba perdiendo durante los 5 últimos días, y además cumplía otra condición
añadida respecto a su sector, también. Pero existía redundancia, puesto que si estaba perdiendo durante los 5 últimos días,
también lo estaría haciendo durante los 3 últimos días. Se eliminó la segunda casuística dejándola comentada en base de datos.

\subsection{Descripción del sistema desarrollado}
\subsubsection{Variables de entrada}
Los ficheros existentes en la carpeta \texttt{data} constituyen la entrada de datos al problema:

\begin{itemize}
 \item \texttt{Analisis.txt}: datos de acciones de empresas del Ibex35. Contiene por este orden, las siguientes columnas:
 nombre de la empresa, precio acción, variación, capitalización, per, rpd, tamaño, \% del Ibex35, per nominal(alto, bajo, medio), rpd nominal,
 sector, variación respecto a 5 días, perdiendo3(si la empresa encadena 3 días de pérdidas), perdiendo5 (si la empresa encadena
 5 días de pérdidas), variación respecto al sector a 5 días, variación respecto al sector a 5 días es menor que -5, variación
 mensual, variación trimestral, variación semestral, variación anual.
 \item \texttt{AnalisisSectores.txt}: datos de sectores del Ibex35. Contiene por este orden, las siguientes columnas:
 nombre del sector, variación, capitalización, per, rpd, \% del Ibex35, perdiendo3, perdiendo5, variación mensual, variación
 trimestral, variación semestral, variación anual.
 \item \texttt{Cartera.txt}: listado de acciones de empresas con su valor.
  La primera línea del fichero tiene el formato:
  \begin{minted}{bash}
    DISPONIBLE XXXX XXXX
  \end{minted}
  
  donde \texttt{XXXX} representa el saldo disponible.
  Las líneas que suceden a ésta tienen el formato:
  
  \begin{minted}{bash}
    ABERTIS_INFR XXXX YYYYY
  \end{minted}

  donde \texttt{XXXX} representa el número de acciones que tenemos de la compañía cuyo nombre encabeza la línea
  y \texttt{YYYY} el valor anterior de esas acciones.
  
 \item \texttt{Noticias.txt}: Las líneas del fichero tienen el formato:
  
  \begin{minted}{bash}
    B.POPULAR Mala 1
  \end{minted} 
 
  donde para la empresa \texttt{B.POPULAR} estamos asegurando que existe una mala noticia con fecha de antig\"uedad 1.
  
  Las antig\"uedades de este fichero habría que actualizarlas a mano, aunque podría automatizarse dicho proceso mediante un
  script que se ejecutara de madrugada, al igual que para recoger los datos de la bolsa.
\end{itemize}

\texttt{bin/main.clp} contiene en un \texttt{deffacts} de nombre \texttt{Entrada} un hecho \texttt{(PrecioDinero 0)}
que contiene el precio del dinero declarado por el BCE. Este valor hay que actualizarlo a mano.

\subsubsection{Variables de salida}
La variable de salida la constituye el fichero \texttt{data/Cartera.txt}, que se sobreescribe con la salida del programa, cuando
en el menú pulsamos la opción correspondiente (c).

\subsubsection{Conocimiento global del sistema}

\subsection{Manual de uso}

El archivo \texttt{bin/scrapeo.py} se debería programar con \texttt{cron} o \texttt{crontab}
para ejecutarse diariamente de madrugada, y que actualizara los datos de la bolsa. En Windows se podría programar su ejecución 
con el programador de tareas del sistema.

Para ejecutar el SE, basta hacer la siguiente sucesión de órdenes(en sistemas UNIX, y situado dentro del directorio \texttt{clp}):

\end{document}
\begin{minted}{bash}
 clips
 (load main.clp)
 (reset)
 (run)
\end{minted}

El menú ofrecerá al usuario opciones, habiéndose imprimido éstas y la cartera con anterioridad. Basta pulsar una opción correcta
(1,2,3,4, o 5) para observar sus resultados. En caso de querer parar la ejecución del sitema, basta introducir una opción
distinta de estas 5.