\documentclass[a4paper,11pt]{article}
% Símbolo del euro
\usepackage[gen]{eurosym}
% Codificación
\usepackage[utf8]{inputenc}
% Idioma
\usepackage[spanish]{babel} % English language/hyphenation
\selectlanguage{spanish}
% Hay que pelearse con babel-spanish para el alineamiento del punto decimal
\decimalpoint
\usepackage{dcolumn}
\newcolumntype{d}[1]{D{.}{\esperiod}{#1}}
\makeatletter
\addto\shorthandsspanish{\let\esperiod\es@period@code}
\makeatother

\usepackage[usenames,dvipsnames]{color} % Coloring code

\usepackage{csvsimple}
\usepackage{adjustbox}
\newsavebox\ltmcbox


% Para matrices
\usepackage{amsmath}
\everymath{\displaystyle}

% Símbolos matemáticos
\usepackage{amssymb}
\let\oldemptyset\emptyset
\let\emptyset\varnothing

% Hipervínculos
\usepackage{url}

\usepackage[section]{placeins} % Para gráficas en su sección.
\usepackage{mathpazo} % Use the Palatino font
\usepackage[T1]{fontenc} % Required for accented characters
\newenvironment{allintypewriter}{\ttfamily}{\par}
\setlength{\parindent}{0pt}
\parskip=8pt
\linespread{1.05} % Change line spacing here, Palatino benefits from a slight increase by default

% Code for bash
\usepackage{minted}

% Imágenes
\usepackage{graphicx}
\usepackage{float}
\usepackage{caption}
\usepackage{wrapfig} % Allows in-line images



% Márgenes
\usepackage{geometry}
 \geometry{
 a4paper,
 total={210mm,297mm},
 left=30mm,
 right=30mm,
 top=25mm,
 bottom=25mm,
 }


% Referencias
\usepackage{fncylab}
\labelformat{figure}{\textit{\figurename\space #1}}

\usepackage{hyperref}
\hypersetup{
  colorlinks   = true, %Colours links instead of ugly boxes
  urlcolor     = blue, %Colour for external hyperlinks
  linkcolor    = blue, %Colour of internal links
  citecolor   = red %Colour of citations
}


\makeatletter
\renewcommand{\@listI}{\itemsep=0pt} % Reduce the space between items in the itemize and enumerate environments and the bibliography
\newcommand{\imagent}[4]{
    \begin{wrapfigure}{#4}{0.7\textwidth}
    \begin{center}
    \includegraphics[width=0.7\textwidth]{#1}
    \end{center}
    \caption{#3}
    \label{#4}
  \end{wrapfigure}
}


\newcommand{\imagen}[4]{
  \begin{minipage}{\linewidth}
    \centering
    \includegraphics[width=#4\textwidth]{#1}
    \captionof{figure}{#2}
    \label{#3}
  \end{minipage}
}

\newcommand{\imgn}[3]{
  \begin{minipage}{\linewidth}
    \centering
    \includegraphics[width=#3\textwidth]{#1}
    \captionof{figure}{#2}
  \end{minipage}
}

% Ejemplo de parámetro: ILS.r
\newcommand{\hrefr}[1]{
\href{../bin/#1}{#1}
}

%Customize enumerate tag
\usepackage{enumitem}
%Sections don't get numbered
%\setcounter{secnumdepth}{0}
\newcommand{\blue}[1]{\textcolor{blue}{#1}}
\begin{document}
%%%%%%%%%%%%%%%%%%%%%%%%%%%%%%%%%%%%%%%%%
% University Assignment Title Page
% LaTeX Template
% Version 1.0 (27/12/12)
%
% This template has been downloaded from:
% http://www.LaTeXTemplates.com
%
% Original author:
% WikiBooks (http://en.wikibooks.org/wiki/LaTeX/Title_Creation)
% Modified by: NCordon (https://github.com/NCordon)
%
% License:
% CC BY-NC-SA 3.0 (http://creativecommons.org/licenses/by-nc-sa/3.0/)
%
% Instructions for using this template:
% This title page is capable of being compiled as is. This is not useful for
% including it in another document. To do this, you have two options:
%
% 1) Copy/paste everything between \begin{document} and \end{document}
% starting at \begin{titlepage} and paste this into another LaTeX file where you
% want your title page.
% OR
% 2) Remove everything outside the \begin{titlepage} and \end{titlepage} and
% move this file to the same directory as the LaTeX file you wish to add it to.
% Then add \input{./title_page_1.tex} to your LaTeX file where you want your
% title page.
%
%%%%%%%%%%%%%%%%%%%%%%%%%%%%%%%%%%%%%%%%%
\begin{titlepage}

\newcommand{\HRule}{\rule{\linewidth}{0.5mm}} % Defines a new command for the horizontal lines, change thickness here

\center % Center everything on the page

%----------------------------------------------------------------------------------------
%	HEADING SECTIONS
%----------------------------------------------------------------------------------------
\textsc{\LARGE Universidad de Granada}\\[1.5cm]
\textsc{\Large Ingeniería del conocimiento}\\[0.5cm]

%----------------------------------------------------------------------------------------
%	TITLE SECTION
%----------------------------------------------------------------------------------------
\bigskip
\HRule \\[0.4cm]
{ \huge \bfseries SE inversor de bolsa}\\[0.4cm] % Title of your document
\HRule \\[1.5cm]

%----------------------------------------------------------------------------------------
%	LOGO SECTION
%----------------------------------------------------------------------------------------

\begin{center}
\includegraphics[width=8cm]{./data/ugr.jpg}
\end{center}
%----------------------------------------------------------------------------------------

\begin{minipage}{\textwidth}
\begin{center} \large
Ignacio Cordón Castillo, 25352973G\\
\url{nachocordon@correo.ugr.es}\\
\ \\
$4^{\circ}$ Doble Grado Matemáticas Informática\\
\end{center}
\end{minipage}


\vspace{\fill}% Fill the rest of the page with whitespace
\large\today
\end{titlepage}

\newpage
\tableofcontents
\newpage
% Examples of inclussion of images
%\imagent{ugr.jpg}{Logo de prueba}{ugr}
%\imagen{ugr.jpg}{Logo de prueba}{ugr2}{size relative to the \textwidth}

\section{Resumen funcionamiento}
La obtención de los datos de la bolsa puede hacerse ejecutando:

\begin{minted}{bash}
 ./scrapeo.py
\end{minted}

disponible en la carpeta \texttt{bin}.

Este archivo se conecta a varias web de interés económico con información disponible sobre la bolsa 
(\href{http://www.bolsamadrid.es}{Bolsa de Madrid}, \href{http://www.eleconomista.es}{El Economista},
\href{http://www.invertia.com/}{Invertia}) para \textit{scrapear} datos de interés (precio por acción,
per, rpd, tamaño de la empresa, variación mensual, trimestral, semestral, anual, capitalización, \ldots) 

Asimismo también efectúa un \textit{scrapeo} de datos de sectores (Bienes, Tecnología, Servicios, \ldots).

Los datos son guardados en los archivos \texttt{Analisis.txt} y \texttt{AnalisisSectores.txt} respectivamente,
disponibles en la carpeta \texttt{data}.

Los archivos de Clips(archivos \texttt{.clp} se encuentran en la carpeta \texttt{clp}.

Por clarificar un poco la estructura de ficheros de la práctica, tenemos:
\begin{itemize}
 \item \textbf{bin}: Archivos de \textit{scrapeo} de datos, y diccionarios con sectores correspondientes a cada empresa 
 del Ibex35 (\texttt{sectores.py}) y denominaciones usadas por cada empresa en las distintas páginas de las que se
 extraen datos (\texttt{empresas.py}).
 \item \textbf{data}: Archivos \texttt{.txt} de entrada al sistema (\texttt{Cartera}, \texttt{Analisis}, 
 \texttt{AnalisisSectores} y \texttt{Noticias}).
 \item \textbf{clp}: Archivos correspondientes a Clips.
\end{itemize}


El sistema carga, por este orden los siguientes módulos, a través del arvhivo \texttt{main.clp}:

\begin{itemize}
  \item \textbf{Módulo de estructura de datos}: carga los templates necesarios para el funcionamiento de la práctica: valores de cartera,
  valores de sectores, noticias, templates para definir valores inestables, peligrosos, infravalorados, sobrevalorados,
  estructura de datos de acciones propuestas \ldots

  \item \textbf{Módulo de lectura, y cálculo de valores inestables}: lee los archivos de cartera, sectores y noticias, y
  efectúa un cálculo de los valores inestables en función de ello.

  \item \textbf{Módulo de cálculo de valores peligrosos}
  
  \item \textbf{Módulo de cálculo de valores sobrevalorados}
  
  \item \textbf{Módulo de cálculo de propuestas}: declara todos los hechos en base de datos correspondientes a cálculo de
  propuestas posibles sobre valores en cartera (para vender), o valores que podrían adquirirse o intercambiarse por alguno en cartera.
  
  \item \textbf{Módulo menú}: muestra al usuario las 5 mejores propuestas con su rendimiento esperado y despliega un menú 
  sobre cuál de dichas propuestas llevar a cabo. En función de la opción escogida, se procesará una venta, una compra, o un
  intercambio de valores (se venden las acciones de una empresa en cartera para adquirir el equivalente en dinero de una
  empresa cuyas acciones son potencialmente revalorizables).
\end{itemize}

Una vez cargado el último módulo se limpian las propuestas calculadas, así como otras variables auxiliares para la ejecución
del menú, y se dispara de nuevo el módulo de cálculo de propuestas hasta que escojamos en el menú una opción distinta a
llevar a cabo alguna de las propuestas.

\section{Proceso de desarrollo}

\subsection{Sesiones con el experto}

\subsection{Procedimiento de validación y verificación}

\end{document}
